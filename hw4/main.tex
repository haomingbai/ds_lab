\documentclass[a4paper]{article}

\usepackage{ctex}

\usepackage{listings}
\usepackage{xcolor}

\definecolor{mygreen}{rgb}{0,0.6,0}
\definecolor{mygray}{rgb}{0.5,0.5,0.5}
\definecolor{myorange}{rgb}{0.9,0.4,0}
\definecolor{mybackground}{rgb}{0.95,0.95,0.92}

\lstdefinestyle{myCstyle}{
  backgroundcolor=\color{mybackground},
  commentstyle=\color{mygreen},
  keywordstyle=\color{blue},
  numberstyle=\tiny\color{mygray},
  stringstyle=\color{myorange},
  basicstyle=\ttfamily\footnotesize,
  breakatwhitespace=false,
  breaklines=true,
  captionpos=b,
  keepspaces=true,
  numbers=left,
  numbersep=5pt,
  showspaces=false,
  showstringspaces=false,
  showtabs=false,
  tabsize=2,
  language=C
}

\lstdefinestyle{myCPPstyle}{
  backgroundcolor=\color{mybackground},
  commentstyle=\color{mygreen},
  keywordstyle=\color{blue},
  numberstyle=\tiny\color{mygray},
  stringstyle=\color{myorange},
  basicstyle=\ttfamily\footnotesize,
  breakatwhitespace=false,
  breaklines=true,
  captionpos=b,
  keepspaces=true,
  numbers=left,
  numbersep=5pt,
  showspaces=false,
  showstringspaces=false,
  showtabs=false,
  tabsize=2,
  language=C++
}


\title{数据结构-第三次作业}

\author{姓名:白昊明, 学号: 2023301350}

\begin{document}
\maketitle
\section*{题目1}

\paragraph*{题目内容} 设计一个算法, 计算一个仅包含字母字符的顺序串s中最大字母出现的次数.

\paragraph*{思路} 最大字母可以通过比较大小获得. 本题采用贪心算法, 当检测到更大字母时将计数器重置为1并且更新待检测字母.

\lstinputlisting[style=myCstyle, caption = {Problem 1: solution}]{./a.c}

\section*{题目2}

\paragraph*{题目内容} 设计一个算法, 判断顺序串s是否为回文 (回文指一个字符串从前向后堵和
从后向前读的结果相同).

\paragraph*{思路} 设置一个栈, 然后在字符串的前半部分(向下取整)进行入栈, 在后半部分出栈. 如果遇到不相同元素则返回否定. 但是本题可以直接分别从首尾遍历字符串, 在速度上恐怕更优化 \dots \dots

此外, 当字符串的长度大于8时, 可以将字符指针强制转化为长整数指针, 然后利用长整数的比较和字符占用相同时钟周期加快运算速度. 另外, 使用C++ 26中即将加入的simd库, 可以充分利用先进CPU的新指令, 实现更快的计算.

\lstinputlisting[style=myCstyle, caption={Problem 2: solution}]{./b.c}

\section*{题目3}
\paragraph*{题目内容} 设有一个顺序串s, 其字符仅由数字和小写字母组成. 设计一个算法, 将s中
的所有数字字符放在前半部分, 将所有小写字母字符放在后半部分, 并给出
算法的时间和空间复杂度.

\paragraph*{思路} 这个题目可以设置两个队列, 一个存放字母, 另一个存放数字. 先遍历字符串, 然后将遍历到的字符送进队列. 完成之后执行出队列操作, 先对存放数字的队列进行出队列, 然后是存放字母的队列. 时间复杂度$O(n)$.

\lstinputlisting[style=myCstyle, caption={Problem 3: solution}]{./c.c}

\section*{题目4}

\paragraph*{题目内容} 如果串中一个长度大于1的子串的全部字符相同, 则称为等值子串. 设计一
个算法, 求顺序串s中一个长度最大的等值子串t, 如果串s中不存在等值子
串, 则t为空串.

\paragraph*{思路}
设置一个字符标志和一个计数器. 当检测到当前字母和下一个字母相同时, 则将计数器置为1并继续遍历,
如果当前字母和字母标志相同, 则将计数器自增.

\lstinputlisting[style=myCppstyle, caption={Problem 4: solution}]{./d.cpp}

\section*{题目5}

\paragraph*{题目内容}
5. 设计一个算法, 删除一个链串s中所有非重叠的“abc”子串. 例如,
s=“aabcabcd”, 删除后s=“ad”.

\paragraph*{思路}
遍历, 遇到匹配立刻删除, 最坏复杂度$O(n * m)$

\lstinputlisting[style=myCppstyle, caption={Problem 5: solution}]{./e.cpp}

\section*{题目6}

\paragraph*{题目内容}
假设字符串s采用链串存储, 设计一个算法, 判断是否为"x@x"形式的串,
其中x是不含"@"的任意串. 例如, 当s="ab@ab"时返回true, 当
s="abcd"时返回false.

\paragraph*{思路}
这个题目可以找到"@"的位置, 然后设置两个指针, 分别指向字符串的开头和'@'之后的下一个字符, 然后进行比较. 如果找到不同的字符就是false, 否则为true.

\lstinputlisting[style=myCppstyle, caption={Problem 6: solution}]{./f.c}

\section*{题目7, 8}

\paragraph*{题目内容}
模式匹配算法

\lstinputlisting[style=myCppstyle, caption={Problem 6: solution}]{./g.c}
\lstinputlisting[style=myCppstyle, caption={Problem 6: solution}]{./h.c}

\end{document}
