\documentclass{article}

\usepackage{listings}
\usepackage{xcolor}

\definecolor{mygreen}{rgb}{0,0.6,0}
\definecolor{mygray}{rgb}{0.5,0.5,0.5}
\definecolor{myorange}{rgb}{0.9,0.4,0}
\definecolor{mybackground}{rgb}{0.95,0.95,0.92}

\lstdefinestyle{myCstyle}{
  backgroundcolor=\color{mybackground},
  commentstyle=\color{mygreen},
  keywordstyle=\color{blue},
  numberstyle=\tiny\color{mygray},
  stringstyle=\color{myorange},
  basicstyle=\ttfamily\footnotesize,
  breakatwhitespace=false,
  breaklines=true,
  captionpos=b,
  keepspaces=true,
  numbers=left,
  numbersep=5pt,
  showspaces=false,
  showstringspaces=false,
  showtabs=false,
  tabsize=2,
  language=C
}

\lstdefinestyle{myCPPstyle}{
  backgroundcolor=\color{mybackground},
  commentstyle=\color{mygreen},
  keywordstyle=\color{blue},
  numberstyle=\tiny\color{mygray},
  stringstyle=\color{myorange},
  basicstyle=\ttfamily\footnotesize,
  breakatwhitespace=false,
  breaklines=true,
  captionpos=b,
  keepspaces=true,
  numbers=left,
  numbersep=5pt,
  showspaces=false,
  showstringspaces=false,
  showtabs=false,
  tabsize=2,
  language=C++
}

\title{Data Structure, HW 5}
\author{Haoming Bai, 2023301350}

\begin{document}
\maketitle
\section*{Problem 1}
Reverse the array for three times. Since we can know that the first element should be positioned at (i + 1) th, then the array should be reversed again around the i + 1 th elem.

\lstinputlisting[style=myCppstyle, caption={Problem 1: solution}]{./a.cpp}

\section*{Problem 2}
Traverse the list, and record the greatest element. If the next element is greater, then update the element's value with index but only update the index when meeting an element as great as the current greatest element.

\lstinputlisting[style=myCppstyle, caption={Problem 2: solution}]{./b.cpp}

\section*{Problem 3}
Modify the value as requested and I cannot unerstand preicsely what can this algorithm actually do.

\lstinputlisting[style=myCppstyle, caption={Problem 3: solution}]{./c.cpp}

\section*{Problem 4}
Transpose with a space complexity of $O(1)$. While the row\_num is not equal to the colume number, then return false, or siwtch the vars which are symmetric.

\lstinputlisting[style=myCppstyle, caption={Problem 4: solution}]{./d.cpp}

\section*{Problem 5}
Add the diagnoals together and remove the central number which is added repeatedly when the row number is odd.

\lstinputlisting[style=myCppstyle, caption={Problem 5: solution}]{./e.cpp}

\end{document}
